%!TEX root = ../thesis.tex
\renewcommand\thechapter{B}
\chapter{Appendix B: Project requirements}
\label{AppendixB}

\section{Notation}
Si farà riferimento ai requisiti secondo le seguenti notazioni:
\begin{itemize}
	\item \textit{O} per i requisiti obbligatori, vincolanti in quanto obiettivo primario richiesto dal committente;
	\item \textit{D} per i requisiti desiderabili, non vincolanti o strettamente necessari,
	ma dal riconoscibile valore aggiunto;
	\item \textit{F} per i requisiti facoltativi, rappresentanti valore aggiunto non strettamente 
	competitivo.
\end{itemize}

Le sigle precedentemente indicate saranno seguite da una coppia sequenziale di numeri, identificativo del requisito.

\section{Original project requirements}

\subsection*{Obiettivi fissati}
Si prevede lo svolgimento dei seguenti obiettivi:
\begin{itemize}
	\item Obbligatori
	\begin{itemize}
		\item \underline{\textit{O01}}: tutti i software del nuovo sistema devono essere correttamente configurati;
		\item \underline{\textit{O02}}: tutti i software del nuovo sistema devono essere in grado di comunicare tra loro;
		\item \underline{\textit{O03}}: deve essere presente una documentazione chiara ed appropriata sia per chi utilizza il software sia per chi lo andrà a mantenere;
	\end{itemize}
	
	\item Desiderabili 
	\begin{itemize}
		\item \underline{\textit{D01}}: possibilità di integrazione con la suite di prodotti Microsoft Office;
		\item \underline{\textit{D02}}: possibilità di effettuare personalizzazioni tramite l'utilizzo di script che andranno ad interagire con i software che compongono il sistema;
	\end{itemize}
	
	\item Facoltativi
	\begin{itemize}
		\item \underline{\textit{F01}}: possibilità di migrazione dei dati relativi a profetti già in produzione sul nuovo sistema;
		\item \underline{\textit{F02}}: possibilità di trasportare il sistema in ambiente di produzione rimpiazzando (anche in maniera graduale) quello già esistente;
	\end{itemize} 
\end{itemize}

\section{Revised project requirements}

\subsection*{Obiettivi fissati}
Si prevede lo svolgimento dei seguenti obiettivi:
\begin{itemize}
	\item Obbligatori
	\begin{itemize}
		\item \underline{\textit{O01}}: tutti i software del nuovo sistema devono essere correttamente configurati;
		\item \underline{\textit{O02}}: tutti i software del nuovo sistema devono essere in grado di comunicare tra loro;
		\item \underline{\textit{O03}}: deve essere presente una documentazione chiara ed appropriata sia per chi utilizza il software sia per chi lo andrà a mantenere;
	\end{itemize}
	
	\item Desiderabili 
	\begin{itemize}
		\item \underline{\textit{D01}}: possibilità di integrazione con la suite di prodotti Microsoft Office;
		\item \underline{\textit{D02}}: possibilità di effettuare personalizzazioni tramite l'utilizzo di script che andranno ad interagire con i software che compongono il sistema;
	\end{itemize}
	
	\item Facoltativi
	\begin{itemize}
		\item \underline{\textit{F01}}: possibilità di migrazione dei dati relativi a profetti già in produzione sul nuovo sistema;
		\item \underline{\textit{F02}}: possibilità di trasportare il sistema in ambiente di produzione rimpiazzando (anche in maniera graduale) quello già esistente;
	\end{itemize} 
\end{itemize}

