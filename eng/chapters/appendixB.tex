%!TEX root = ../thesis.tex
\renewcommand\thechapter{B}
\chapter{Appendix B: Project requirements}
\label{AppendixB}

\section{Notation}
	Requirements in this text are referenced as:
	\begin{itemize}
		\item \textit{O} (\Quote{Obbligatorio} in italian) for Mandatory: that have to be implemented, represent the core of the project
		\item \textit{D} (Desiderabile in italian) for Desirable: not mandatory for the final objective but add greater value
		\item \textit{F} (Facoltativi in italian) for Optional: add value to the project but not as much as the previous ones, carried out only if there is time left for them
	\end{itemize}
	Each requirement has a unique ID that is composed by the first letter of their importance (from the Italian word to resemble the \Quote{Piano di Lavoro}) and an increasing number.

\newpage
\section{Original project requirements}
	The requirements that have been agreed with the tutor and that are included in the Piano di Lavoro are:
	\begin{itemize}
		\item Mandatory
		\begin{itemize}
			\item \underline{\textit{O01}}: all the new software must be correctly configured
			\item \underline{\textit{O02}}: Jira and Confluence must be fully interconnected
			\item \underline{\textit{O03}}: it is required a clear and suitable documentation for both the users and the maintainers
		\end{itemize}
		\item Desirable
		\begin{itemize}
			\item \underline{\textit{D01}}: integration with Microsoft Office
			\item \underline{\textit{D02}}: create scripts that will interact with the software that compose the environmentX
		\end{itemize}
		\item Optional
		\begin{itemize}
			\item \underline{\textit{F01}}: migrate existing data from the old system to the new
			\item \underline{\textit{F02}}: migrate the system to the production environment to replace (even gradually) the existing one
		\end{itemize} 
	\end{itemize}

\newpage
\section{Revised project requirements}
	Here is a list of the requirements that have been revised during the internship period:
	\begin{itemize}
		\item Mandatory
		\begin{itemize}
			\item \underline{\textit{O01}}: all the new software must be correctly configured
			\item \underline{\textit{O02}}: Jira and Confluence must be fully interconnected
			\item \underline{\textit{O03}}: it is required a clear and suitable documentation for both the users and the maintainers
			\item \underline{\textit{O04}}: the new system must be able to connect with the repository hosted in GitLab
			\item \underline{\textit{O05}}: there must be a live product roadmap that can be available for various members of the company
		\end{itemize}
		\item Desirable
		\begin{itemize}
			\item \underline{\textit{D01}}: integration with Microsoft Office
			\item \underline{\textit{D02}}: create scripts that will interact with the software that compose the environment
			\item \underline{\textit{D03}}: customize the interface with the logo and the colors of the company
			\item \underline{\textit{D04}}: let client connect only via HTTPS protocol
			\item \underline{\textit{D05}}: having a customizable interface for inserting issues
		\end{itemize}
		\item Optional
		\begin{itemize}
			\item \underline{\textit{F01}}: migrate existing data from the old system to the new
			\item \underline{\textit{F02}}: migrate the system to the production environment to replace (even gradually) the existing one
			\item \underline{\textit{F03}}: create templates for documents in Confluence
		\end{itemize} 
	\end{itemize}