%!TEX root = ../thesis.tex
\begin{savequote}[75mm]
Nulla facilisi. In vel sem. Morbi id urna in diam dignissim feugiat. Proin molestie tortor eu velit. Aliquam erat volutpat. Nullam ultrices, diam tempus vulputate egestas, eros pede varius leo.
\qauthor{Quoteauthor Lastname}
\end{savequote}

\chapter{Descrizione del progetto di stage}

%\newthought{There's something to be said} for having a good opening line. Morbi commodo, ipsum sed pharetra gravida, orci  $x = 1/\alpha$ magna rhoncus neque, id pulvinar odio lorem non turpis \cite{Eigen1971, Knuth1968}. Nullam sit amet enim. Suspendisse id velit vitae ligula volutpat condimentum. Aliquam erat volutpat. Sed quis velit. Nulla facilisi. Nulla libero. Vivamus pharetra posuere sapien. Nam consectetuer. Sed aliquam, nunc eget euismod ullamcorper, lectus nunc ullamcorper orci, fermentum bibendum enim nibh eget ipsum. Donec porttitor ligula eu dolor. Maecenas vitae nulla consequat libero cursus venenatis. Nam magna enim, accumsan eu, blandit sed, blandit a, eros.
%$$\zeta = \frac{1039}{\pi}$$


% For an example of a full page figure, see Fig.~\ref{fig:myFullPageFigure}.

\section{The company's needs}
che cosa sta utilizzando athonet adesso per la gestione del lavoro e del tracking\\
come issue tracking system, gestore di wiki interno, ecc --> tanti tool scorrelati tra loro\\
perchè athonet ha la necessità di utilizzare tool differenti \\
(+ stabili, meglio documentati, miglior UI / UX etc.)

\section{Planning}
in base a cosa ho pianificato
come ho parlato con il tutor per fare capire i loro bisogni, lo scopo finale, e come arrivarci
step da fare\\
parlare poi di chi avrà bisogno del tool e con chi interagirò durante lo stage\\
fase preliminare di pianificazione insieme ad altre figure (aree diverse, managemeng, product ownership...)

\section{come ho approciato il problema}

\section{riassunto piano di lavoro con grafici}
preventivo ore + lavoro\\
risultato del paragrafo precedente
