%!TEX root = ../thesis.tex
\begin{savequote}[75mm]
If a team couldn’t be fed with two pizzas, it was too big.
\qauthor{Jeff Bezos}
\end{savequote}

%https://areomagazine.com/2019/04/10/agile-and-the-software-industrys-ideology-problem/ 
%https://www.softwaretestinghelp.com/agile-manifesto/
%https://www.smartsheet.com/comprehensive-guide-values-principles-agile-manifesto

\chapter{Agile Software Development}
\label{chapter_3}

Before getting into the implementation and adaptation of Jira and Confluence, let's take a step back and understand the concept of Agile, a \Quote{Software Development Life Cycle} (or SDLC) model.\\
SDLCs did not emerge until the 1960s; they are considered to be the oldest formalization of \gls{framework}\glsadd{Framework}.
A SDLC refers to the ensemble of activities that compose a software project.\\
It starts with the concepts of understanding the problem as well as the requirements and it ends with the retirement of the system (when there is no more maintenance) or with the cancellation of the project.\\
Small projects (generally for a single person) have a simpler life cycle: find the problem and write a program to solve it; once the problem has been solved, the program can be deleted and forgotten.\\
In larger projects, that require a team to be developed, there must be some explicit rules to set a higher quality for the software.\\
As activities are assigned to different people, it becomes critical that all participants share a common view of the execution of the project.
A SDLC model is a framework providing the ordering and dependencies of life cycle activities, managing these can be a major impact for a successful project and its duration.\\
For example, a change in requirements during implementation may invalidate a substantial amount of work and delay the delivery of the system by several months.
Different life cycle models prescribe different actions to handle such changes\cite{software_lyfe_cycle}.\\
There are many SDLCs like \Quote{Waterfall}, \Quote{Prototyping}, \Quote{Iterative} and \Quote{Incremental Development}, \Quote{Spiral Development}, \Quote{Rapid Application Development}, and \Quote{Extreme Programming} (XP). 
\begin{figure}[H]
	\centering
	\includegraphics[width=.7\textwidth]{resources/prototype}\\
	\caption{The \Quote{Prototype} model}
\end{figure}
\begin{figure}[H]
	\centering
	\includegraphics[width=.8\textwidth]{resources/warterfall}\\
	\caption{The \Quote{Waterfall} model}
\end{figure}
%https://www.iso.org/obp/ui/#iso:std:iso-iec:tr:24774:ed-2:v1:en
%todo citare https://www.iso.org/standard/53815.html !!
As the word \Quote{model} suggests, each company can apply its own SDLC designed ad hoc for their internal use.\\
This led to the creation of a standard that presents the guidelines for elements that are most frequently used in describing a process: title, purpose, outcomes, activities, task and information item.
%todo citare
The \Quote{ISO/IEC TR 24774:2010: Systems and software engineering -- Life cycle management -- Guidelines for process description}\cite{iso_53815}.\\
The complexity and slowness in producing a concrete product in older SDLCs brought the need for a faster and more communicative model.
\begin{figure}[H]
	\centering
	\includegraphics[width=\textwidth]{resources/BffFTn_CAAAEvGn}\\
	\caption{Dilbert on creating a framework}
\end{figure}
The Agile SDLC model is a combination of \Quote{iterative} and \Quote{incremental} process models with focus on process adaptability and customer satisfaction by rapidly and continuously deliver of working software product\cite{sdlc_agile_model}.\\
This chapter describes the most fundamental points of the Agile method, how it started and the adaptations that derived from it, like Scrum.
At the end, it also explains how Athonet's adaptation of the Agile life cycle works.

\section{The need for a new Software Life Cycle}
	The decade of 1990 represented a very important turning point for the digital industry.
	Computers were spreading everywhere and the software companies faced the so-called application development crisis.\\
	The problem was that businesses moved too fast and within the space of three years, requirements, systems, and even entire businesses were likely to change. 
	It meant that a lot of software ended up being incomplete or canceled halfway and the ones that made it through, even if they fulfilled the original objectives of the client, may not meet all the business needs\cite{agility-beyond-history}.
	SDLC models can be of two types:
	\begin{itemize}
		\item \textbf{Iterative}: enhances the evolving versions until the complete system is implemented and ready to be deployed
		\item \textbf{Incremental}: the product is designed, implemented and tested incrementally until it is finished
	\end{itemize}
	One of the most characteristic models used before Agile was the Sequential model (or Waterfall).
	The Waterfall model became very famous because it has many strong points as:
	\begin{itemize}
		\item Uses clear structure
		\item Determines the end goal early
		\item Transfers information well
	\end{itemize}
	On the other hand, it became obsolete when projects started to be more dynamic and complex.
	Some of it's disadvantages in modern software projects are:
	\begin{itemize}
		\item Makes changes difficult
		\item Excludes the client and/or end user
		\item Delays testing until after completion
	\end{itemize}
	The excessive documentation, the forceful binding to the unchangeable decisions made early in the project and the little communication with the client brought to the need of a new model that prioritizes the product and the stakeholders over bureaucracy.\\
	Those things frustrated people like Jon Kern, an aerospace engineer in the 1990s that with other figures from different industries \Quote{were looking for something that was more timely and responsive}, as he noted.
	He was one of 17 software leaders that started meeting informally and talking about ways to develop software in a simpler way without the excess of documentation and other strict rules.\\
	These talks led to the now famous \Quote{Snowbird} meeting (in Utah, February 2001), when the \Quote{Agile Manifesto} was written down and published.

\section{The Agile manifesto}
	The Agile Manifesto is a brief document built on four foundational values and twelve supporting principles for Agile software development\cite{agilemanifesto}.\\
	The four values written on the official website\cite{agile_official} are:
	\begin{itemize}
		\item \textbf{Individuals and interactions} over processes and tools
		\item \textbf{Working software} over comprehensive documentation
		\item \textbf{Customer collaboration} over contract negotiation
		\item \textbf{Responding to change} over following a plan
	\end{itemize}
	The responsiveness of people and embracing the importance of changes are the fundamentals of Agile.
	Although documentation is secondary, it's important to note that Agile streamlines documentation and does not eliminate it.\\
	These twelve principles emphasize things like \Quote{early and continuous delivery of valuable software} and \Quote{continuous attention to technical excellence}, which are: 
	\begin{enumerate}
		\item Our highest priority is to satisfy the customer through early and continuous delivery of valuable software.
		\item Welcome changing requirements, even late in development. Agile processes harness change for the customer's competitive advantage.
		\item Deliver working software frequently, from a couple of weeks to a couple of months, with a preference to the shorter timescale.
		\item Business people and developers must work together daily throughout the project.	
		\item Build projects around motivated individuals. Give them the environment and support they need, and trust them to get the job done.
		\item The most efficient and effective method of conveying information to and within a development team is face-to-face conversation.
		\item Working software is the primary measure of progress.
		\item Agile processes promote sustainable development. The sponsors, developers, and users should be able to maintain a constant pace indefinitely.	
		\item Continuous attention to technical excellence and good design enhances agility.
		\item Simplicity--the art of maximizing the amount of work not done--is essential.
		\item The best architectures, requirements, and designs emerge from self-organizing teams.
		\item At regular intervals, the team reflects on how to become more effective, then tunes and adjusts its behavior accordingly.
	\end{enumerate}
	All of them are important but, in my opinion, the ones that add the most value to the Agile line of thought and that differentiate it from other methods are the second, the fourth and the sixth: they represent the intent of placing the product and the customer above everything else, allowing the use of small informal meetings (even if the decisions should be recorded) and the easy change of requirements because the client is always involved, even as an end user (tester).\\
	Each Agile methodology applies the four values in different ways.
	However, all of them rely on these values to guide the development and delivery of high-quality, working software\cite{4-values-of-the-agile-manifesto}.

\section{Agile's little big cousins}
	While Agile's manifesto contains values and principles, these are not prescriptive.
	In fact the manifesto does not outline specific processes, procedures or best practices.
	The goal was not to develop a rigid framework but rather create a mindset for software development.
	Agile is a blanket term that describes a set of software development principles.\\
	There are many methodologies that derive from Agile's thinking, the most famous ones, according to the annual survey\cite{state-of-agile} from VersionOne's team are:
	\begin{itemize}
		\item Scrum
		\item Scrumban
		\item Kanban
		\item Extreme Programming (XP)
	\end{itemize}
	The essence of Scrum is being \Quote{agile} as in fast: having a small team that is highly flexible and adaptive.
	\begin{figure}[H]
		\centering
		\includegraphics[width=.8\textwidth]{resources/agile-usage-chart}\\
		\caption{Percentage of the use of Agile derived methods according to VersionOne}
	\end{figure}
	This survey is quite interesting because it provides information from small and big companies that want to share their experience.
	A very important fact to be noted is that although technology companies are the ones that have participated the most to the survey, there are other industries that use Agile and are interested in sharing their experience.\\
	In the following figure there are some percentages that indicate some areas of the companies that participated to VersionOne's survey.
	\begin{figure}[H]
		\centering
		\includegraphics[width=.8\textwidth]{resources/Untitled_2}\\
		\caption{Areas of some of the companies that participated to VersionOne's survey}
	\end{figure}
	According to the previous figure\cite{state-of-agile}, only 25\% of the companies work in the technology business.\\
	This survey states that for the year 2018 (which marked their 13th annual report) the reasons for adopting Agile were productivity, improving team morale, reducing product risk (with a lesser percentage than the previous year) and reducing project costs.\\
	The measures of success mostly cited by the respondents were customer, or user, satisfaction and business value.\\
	According to these companies, some of the most indicated benefits of adopting Agile are:
	\begin{figure}[H]
		\centering
		\includegraphics[width=.9\textwidth]{resources/Untitled}\\
		\caption{Some of the most voted benefits of adopting an Agile methodology}
	\end{figure}
	According to the previous figure\cite{state-of-agile}, 74\% of the companies the Agile methodology accelerates software delivery.\\
	Also the questionnaire contained a question about what are the recommended Agile project management tools.
	\begin{figure}[H]
		\centering
		\includegraphics[width=.9\textwidth]{resources/Screenshot}\\
		\caption{Some of the most recommended Agile management tools}
	\end{figure}
	As we can see Jira is the second most recommended one.

\section{Agile in practice}
	Let's introduce the methodologies that two large companies, Spotify and Amazon, have derived from Agile's line of thought.
	These businesses succeeded using their own ad hoc methods, and now they have chosen to document and release them for others to use.
	\subsubsection{Spotify}
		Spotify has become the most popular music streaming application.
		It was launched in 2008 and now has more than 1600 employees.
		The success of this company is rooted in their Agile adaptation: the \Quote{Spotify Tribe Engineering Model}.\\\\
		This is composed by:\\\\
		\begin{minipage}[c]{0.25\textwidth}
			\begin{itemize}
				\item Squads
				\item Tribes
				\item Chapter
				\item Guild
				\item Trio
				\item Alliance
				\item Chief Architect
			\end{itemize}
		\end{minipage}
		\hfill
		\begin{minipage}[c]{0.70\textwidth}
			\vspace{-2cm}
			\begin{figure}[H]
				\centering
				\includegraphics[width=\textwidth]{resources/spot}\\
				\caption{The Waterfall and Prototype SDLC models}
			\end{figure}
		\end{minipage}\\
		These groups and roles are very coherent because the bedrock of the Spotify Tribe model is autonomy and trust\cite{exploring-key-elements-of-spotifys}.
		This methodology is introduced on Spotify's lab website as \Quote{Spotify engineering culture}\cite{spotify-engineering-culture}.
		
	\subsubsection{Amazon}
		Amazon's organization can be seen unusual among other large firms.
		This company is characterized an ubiquitous \Quote{customer-obsessed} mindset.
		Everyone is expected to know about the data that is generated by customers so that Amazon can enhance the impact of what it offers.\\		
		The methodology adopted by Amazon is described, like Agile's manifesto, in 14 \Quote{leadership principles} on their jobs website\cite{amazon_principles}.\\
		One of the most famous quotes by Jeff Bezos, Amazon's CEO is \Quote{if a team couldn’t be fed with two pizzas, it was too big}.\\
		As Bezos explained: \Quote{The Two-Pizza Team is autonomous. Interaction with other teams is limited, and when it does occur, it is well documented, and interfaces are clearly defined. [...] One of the primary goals is to lower the communications overhead in organizations, including the number of meetings, coordination points, planning, testing, or releases. Teams that are more independent move faster}\cite{how-amazon-became-agile}.

\section{The roles in Agile}
	A distinctive characteristic of the Agile methodology is it's definition of roles: they are not fixed positions, any given person can take on one or more roles and switch over time, also any given role may have zero or more people in it at any given point in a project\cite{agileRoles}.
	The ideal team is considered to be composed of five or six people.\\
	These roles are:
	\begin{itemize}
		\item \textbf{Team leader}: team coach or project lead in other methods (Scrum-master e.g.), he is responsible for facilitating the team, obtaining resources for it, and protecting it from problems
		\item \textbf{Product owner}: an executive or key stakeholder, the Product Owner has a vision for the end product and a sense of how it will fit into the company’s long-term goal
		\item \textbf{Team member}: developer or programmer, is responsible for the creation and delivery of a system
		\item \textbf{Stakeholder}: any other person that has direct or indirect interest in the project
	\end{itemize}

\section{Time cycles and metrics}
	Scrum teams coordinate development into time-boxed \Quote{Sprints}.
	Outside the Sprint, teams organize and estimate the amount of work that can be concluded.
	The goal is to have all the forecasted work completed by the end of the sprint.\\
	Metrics are used to track the completion of tasks, these are called \Quote{burndown reports} and are graphs where the X-axis represents time, and the Y-axis refers to the amount of work left to complete, measured in either story points or hours.
	\begin{figure}[H]
		\centering
		\includegraphics[width=.95\textwidth]{resources/burndown}\\
		\caption{Example of \Quote{burndown report}}
	\end{figure}
	The Sprint is one of the most important structural bricks in Agile, the other principal ones, according to Atlassian's \Quote{Agile Coach}\cite{epics-stories-themes}, are:
	\begin{itemize}
		\item \textbf{Stories}: short requirements or requests written from the perspective of an end user
		\item \textbf{Epics}: large bodies of work that can be broken down into a number of smaller tasks (called stories)
		\item \textbf{Initiatives}: collections of epics that drive toward a common goal
		\item \textbf{Themes}: large focus areas that span the organization
	\end{itemize}
	\vspace{-.5cm}
	\begin{figure}[H]
		\centering
		\includegraphics[width=.8\textwidth]{resources/Themes}\\
		\caption{Structural bricks in Agile according to Atlassian's Agile Coach}
	\end{figure}

\section{Disadvantages of Agile Software Development}
	Despite the benefits offered by the Agile model, transition a company's way of working to it is not that easy and if done wrong it may make damage instead of good.
	According to the American entrepreneur Adam Fridman, here are the drawbacks\cite{massive-downside-of-agile} of Agile:
	\begin{enumerate}
		\item \textbf{Less predictability}: developers may not be able to quantify the full extend of the required effort
		\item \textbf{More time and commitment}: a constant interaction, with many face-to-face conversations, is required
		\item \textbf{Greater demands on developers and clients}: extensive user involvement that impacts the quality and success of the project
		\item \textbf{Lack of necessary documentation}: new members that join the team may need more time to understand the project
		\item \textbf{Project easily falls off track}: if a consumer's feedback or communications are not clear, a developer might focus on the wrong areas of development
	\end{enumerate}
	
\section{What Agile variant Athonet uses}
	As many small companies, Athonet will not strictly use one Agile implementation.
	This is also due because of the nature of their product that is not released like simple software patches but it is presented in a more monolithic version.
	After a research for the available software development life-cycle tools on the market, they chose to try Atlassian's Jira in tandem with Confluence.\\
	As I will say in \Chapref{chapter_5}, the managers liked the idea of Kanban because it allows the employees to \Quote{come in the morning and choose what they want to work on from the backlog} without giving them a two week period of time but letting them complete the tasks in time for the following release.
	But they also liked the idea of having a tool that could transition from a type of project to another in case they want to start applying stricter rules in the future.
	\begin{figure}[H]
		\centering
		\includegraphics[width=\textwidth]{resources/Dilbert_Training_Agile_Programming}\\
		\caption{The Waterfall and Prototype SDLC models}
	\end{figure}

