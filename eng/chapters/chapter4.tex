%!TEX root = ../dissertation.tex
\begin{savequote}[75mm]
Nulla facilisi. In vel sem. Morbi id urna in diam dignissim feugiat. Proin molestie tortor eu velit. Aliquam erat volutpat. Nullam ultrices, diam tempus vulputate egestas, eros pede varius leo.
\qauthor{Quoteauthor Lastname}
\end{savequote}

\chapter{Lo stage}

questo capitolo rappresenta il corpo di questo documento e contiene il lavoro svolto durante lo stage suddiviso in macro periodi

\section{implementazione (installazione e configurazione iniziale)}

	\subsection{strumenti di lavoro}
		con cosa lavoro (hw e sw)\\
		installazione dei software (riferimento a capitolo precedente)\\
		ho lavorato su una vm centos con queste caratteristiche ...
		
	\subsection{cosa ho notato che serviva installare insieme}
		database\\
		mi ha fatto ritardare rispetto al piano di lavoro? no, mi ero aspettato ci fossero software terzi da configurare / imparare
	
	\subsection{configurazione iniziale dei tool}
		interconnessione tra i tool
	
	\subsection{prima variazione dei requisiti in corso d'opera}
		non usare bitbucket ma gitlab\\
		visto il grosso ammontare di elementi customizzabili è stato necessario scremare le cose e capire cosa si poteva facilmente aggiungere e cosa lasciare per dopo\\
		abbellimento dell'environment
	
	\subsection{primo periodo di "gioco" / capire i prodotti "in action"}
		creazione di progetti di mock\\
		interconnetterli\\
		capire il workflow delle issue\\
		utilizzare gitlab (con account personale su server aziendale e progettini di mock) per effettuare transizioni automatiche delle issue (spiegare correlazione tra progetti in gitlab e in jira)
	
	\subsection{personalizzazione interfaccia grafica}
		a causa della poca disponibilità in questo primo periodo di marco e paolo che utilizzeranno questo tool in maniera intensiva rispetto al tutor, ho fatto un task secondario come quello della personalizzazione dell'interfaccia grafica
	
	\subsection{snapshot della macchina per salvare il lavoro svolto per ora}
		parlare di milestone / baseline\\
		come le ho pensate nel piano di lavoro

\section{primi progetti di mock più veritieri e primi feedback}

	\subsection{progetti di mock}
		idee del tutor\\
		prime demo con lui per capire se questo tool effettivamente copre le necessità di base dell'azienda
		
	\subsection{integrazione effettiva con gitlab}
		
		in questo periodo vista la scarista di opzioni di gitla nativo si è scelto di usare un plugin\\
		(decontestualizzare il tempo, a posteriori, ragionando per milestone)\\
		visto integrazione nativa\\
		scelta di utilizzare un plugin\\
		costa ma è migliore (descrivere da quale punto di vista)
	
	\subsection{prime riunioni "serie" e cosa ne è uscito}
		primo feedback e discussioni di come può evolvere il progetto e come può essere applicato ai loro workflow\\
		riflessioni personali: a questo punto sto rispettando il piano di lavoro iniziale? sono in ritardo / anticipo?
	
	\subsection{nuovo cambio di requisiti}
		giustificare --> dopo fase di studio / riscontro\\
		cosa può essere implementato subito, cosa no, come viene usato\\
		campi e workflow custom\\
		mapping tra processi jira e interni (sprint)
	
	\subsection{documentazione}
		scrittura della bozza di documentazione e passaggio della documentazione in confluence
		
	\subsection{nuovo snapshot della macchina}
		nuova baseline / milestone

\section{passaggio in produzione}

	After the approval for using the tools by other departments (R\&D) it's time to transition it / move it to production
	
	\subsection{migrazione da redmine}
		tool automatico di migrazione
		collegamento con redmine, lo fa in maniera automatica
		e se va male? c'è sempre lo snapshot
	
	\subsection{primi progetti non di mock (con importanza effettiva [anche se minima] ed in corso)}
	
	\subsection{fine tuning del progetto / prodotto finale}
		interazione con le persone
		in base alle necessità degli utenti e di come lo usano faccio minime modifiche in produzione
		miglioramento della documentazione
	
	\subsection{come viene utilizzato questo tool}
		è veramente agile? 
		è un dialetto?
		è un misto?
		perchè athonet lo sta usando in questo modo?

\section{feedback finale e cosa vorrebbero che venga implementato in futuro}

	\subsection{feedback finale}
		feedback da parte del tutor
		
		feedback da responsabile strategia aziendale (gl)
		feedback da responsabile del prodotto (aka product ownner / hesham)
		feedback da responsabile sviluppo + testing
			
		feedback da parte di tutti gli utenti
	
	\subsection{quali piani ha athonet in futuro per questo tool}
		arrivare ad utilizzare agile in maniera rigida?
		continuare a fare misto?
