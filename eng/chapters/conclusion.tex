%!TEX root = ../thesis.tex
\chapter{Conclusions}
\label{conclusions}

After having explained how this solution has been implemented, I would like to draw the conclusions on the project, internship experience and knowledge acquired.

\section{Improvement and future implementations}
	As many other projects there are many improvements that can applied.
	For the one presented in this thesis, the most important improvement is to deploy it in an environment such that it could become more portable and easier to operate in case of maintenance.\\
	This would involve separating the database from the VM and storing it into separated machines which would have dedicated disks arranged in a redundant way.
	This would add a layer of security, over regular backups, in a disaster recovery plan.\\
	Another important improvement would be separating the Confluence instance from the Jira one and, instead of having them running on a dedicated VM each, creating a Docker container so that they would be easier to handle (copy, migrate, update, etc.).
	The use of containers would allow a better usage of the host machine's resources like memory and CPU power by sharing them.
	These would otherwise be split and left unused by the VMs.
	This kind of improvements should be implemented when planning the migration of the services in a production environment.
	Other future improvements can be installing plugins that would allow connectivity with the Microsoft Office suite allowing employees to produce documents and reports without learning how to do it in Confluence and just uploading them.

\section{Final Gantt diagram}
%	come si discosta da quello iniziale\\
%	cosa ha causato questo discostamento\\
%	ho pianificato male
%	
\section{Objectives achievement}

\section{What I have learned}

	During this internship I had the possibility to better understand the hierarchy of a company.
	This argument was well introduced in the Software Engineering course, where the focus was on what are the various roles in a software company and how they operate.\\
	In Athonet I had the possibility to see this firs hand and to interact with people on more levels of responsibility, from the CTO, to the product owner to the managers of verification and development teams.
	
	About the software, it is important to know how to use software like this not only for tracking the status of issues but on how to see other information about them as well.
	
	
	

\section{Personal considerations}

	In a growing company like Athonet it is useful to set some boundaries, not only for lower level employees like developers, but for the management as well.
	It is important to understand how 
	
	
	In Athonet a software like this 
	
	It will be very difficult for them to implement an agile filosofia all'interno di tutta l'azienda

	

%	in un'azienda in crescita è molto utile darsi dei paletti\\
%	in athonet funzionerà una cosa del genere?  software monolitico / complesso (service oriented architecture)\\
%	sta funzionando questo tool adesso per il breve tempo che l'ho visto io in produzione?
%	valore aggiunto all'azienda, cosa vede il cliente

%	Quanto i corsi universitari mi abbiano preparato ed aiutato ad affrontare uno stage di questo tipo