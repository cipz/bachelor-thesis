%!TEX root = ../thesis.tex
\chapter{Introduction}
\label{introduction}

introduzione a significato di way of working (in SW)
	cercare articoli da blog per spiegare
	martin fowler		

fare tracking delle isssue / bug è diventato difficile / complesso
	molti tool open source che fanno anche documentazione e code review (altri aspetti)

l'evoluzione negli ultimi anni nelle aziende IT
	necessità di organizzazione delle azienda e la necessità di avere una gerarchia (o simil gerarchia)

ho sperimentato questo approccio in athonet, mostrata interessata all'utilizzo di un gestionale sw di tipo agile per la gestione dei processi di sviluppo sw interni

I have sperimented ... 

\section{Premise}

	Questo documento rappresenta la tesi e il report dello stage conclusivo al percorso di laurea ...
	
	contiene la descrizione dello stage curricolare + introduzione all'ambito in cui è stato fatto, insieme ad un'introduzione dell'argomento geerale, in questo caso la metodologia Agile...

	Per introdurre alcuni concetti della metodologia Agile e come intercalare per facilitare la lettura del documento, verranno utilizzati comic strip di Dilbert, disegnato da Scott Adams, un famoso ...
	
\section{The company}
	breve paragrafo in cui descrivo l'azienda
	quando è stata fondata
	con che visione
	di cosa si occupa
	scrivere che ha vinto tot premi e per cosa (link ad articolo)
	dove sono adesso (in via di crescita e sviluppo)
	cosa pensano di fare di buono nel mondo e per il futuro
	
	Athonet nasce dall'idea di Gianluca e Karim che hanno saputo vedere oltre la tecnologia del momento proiettandosi sul futuro ....
	
	\begin{figure}[H]
		\centering
		\includegraphics[width=.7\textwidth]{resources/ath_logo}\\
		\caption{Athonet's logo}
	\end{figure}
		

\section{The project}
	cosa mi ha portato a scegliere questo stage rispetto ad altri
	breve descrizione del progetto, da riprendere successivamente

\section{Organization}
	This document is organized as follows:
	\begin{itemize}
		\item Chapter 1 or \textit{Introduction}: describes the overall content of this document
		\item Chapter 2 or \textit{The internship project}: describes in detail the objectives and planning of the internship project
		\item Chapter 3 or \textit{Agile processes and methodologies}: an introduction to the Agile software development
		\item Chapter 4 or \textit{Jira and Confluence: the essentials}: describes the functionalities of Jira and Confluence
		\item Chapter 5 or \textit{Project implementation}: details how the project has been implemented by dividing it into time periods
		\item Chapter 6 or \textit{Conclusions}: contains the retrospective of the project, future developments and personal considerations
	\end{itemize}
