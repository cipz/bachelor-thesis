%!TEX root = ../dissertation.tex
% the abstract

%\hl{la crescita in base al numero di dipendenti necessità di avere un tool complesso e sofisticato per gestione della parte di sviluppo sw
%non è il solito gestionale, ma sono tool specifici che considerano i trend a livello di sviluppo
%
%raccontare risultato ottenuto dello stage
%il lavoro di questa tesi è stato ... tool più conosciuto dal mercato ... installare / config ... 
%ottenere approvazione da management .. migrazione dei sistemi in uso nel nuovo gestionale}

\hl{ABSTRACT}

%Il presente documento descrive il lavoro svolto durante il periodo di stage, della durata
%di circa trecentoquaranta ore, presso l’azienda Siav S.p.a. Lo scopo principale dello
%stage è stato quello di sviluppare un software di gestione dei KPI (Key Performace
%Indicator), ossia delle metriche che misurano l’efficacia dei processi aziendali. Tali KPI
%devono essere calcolati su dei log che arrivano direttamente dai MES delle aziende
%utenti dell’applicazione. Le caratteristiche richieste dall’azienda proponente erano
%innanzitutto che il prodotto fosse idoneo a girare su cloud ed in secondo luogo che
%fosse costituito da un’architettura a microservizi e sviluppato con framework di ultima
%generazione.
%Come obiettivo principale ed obbligatorio l’azienda ha richiesto che nella durata
%dello stage venisse sviluppata la parte di backend dell’applicazione, che comprendeva
%principalmente un metodo di storage dei log parziali che arrivavano dai MES, un
%metodo di storicizzazione dei log e lo sviluppo della libreria di calcolo dei KPI.
%Come obiettivo facoltativo di tale stave è stato inserito quello di sviluppare un
%Proof of Concept del frontend, con il fine di permettere un utilizzo dimostrativo
%dell’applicazione.