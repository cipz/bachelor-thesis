%----------------------------------------------------------------------------------------
%   USEFUL COMMANDS
%----------------------------------------------------------------------------------------

\newcommand{\dipartimento}{Dipartimento di Matematica ``Tullio Levi-Civita''}

%----------------------------------------------------------------------------------------
% 	USER DATA
%----------------------------------------------------------------------------------------

% Data di approvazione del piano da parte del tutor interno; nel formato GG Mese AAAA
% compilare inserendo al posto di GG 2 cifre per il giorno, e al posto di 
% AAAA 4 cifre per l'anno
\newcommand{\dataApprovazione}{\today}

% Dati dello Studente
\newcommand{\nomeStudente}{Stefan Ciprian}
\newcommand{\cognomeStudente}{Voinea}
\newcommand{\matricolaStudente}{1143057}
\newcommand{\emailStudente}{stefanciprian.voinea@studenti.unipd.it}
\newcommand{\telStudente}{+39 328 179 6441}

% Dati del Tutor Aziendale
\newcommand{\nomeTutorAziendale}{Fabio}
\newcommand{\cognomeTutorAziendale}{Giust}
\newcommand{\emailTutorAziendale}{fabio.giust@athonet.com}
\newcommand{\telTutorAziendale}{+39 3497589727}
\newcommand{\ruoloTutorAziendale}{Senior System Architect}

% Dati dell'Azienda
\newcommand{\ragioneSocAzienda}{Athonet S.R.L.}
\newcommand{\indirizzoAzienda}{ Via Cà del Luogo, 8, 36050 Bolzano vicentino (VI)}
\newcommand{\sitoAzienda}{https://www.athonet.com/}
\newcommand{\emailAzienda}{mail@mail.it}
\newcommand{\partitaIVAAzienda}{P.IVA 12345678999}

% Dati del Tutor Interno (Docente)
\newcommand{\titoloTutorInterno}{Prof.}
\newcommand{\nomeTutorInterno}{Armir}
\newcommand{\cognomeTutorInterno}{Bujari}

\newcommand{\prospettoSettimanale}{
     % Personalizzare indicando in lista, i vari task settimana per settimana
     % sostituire a XX il totale ore della settimana
    \begin{itemize}
		\item \textbf{Prima Settimana (40 ore)}
		\begin{itemize}
			\item Studio della suite Atlassian
			\item Inizio della configurazione dell'ambiente di lavoro e dell'infrastruttura
		\end{itemize}
		\item \textbf{Seconda Settimana (40 ore)} 
		\begin{itemize}
			\item Studio della suite Atlassian
			\item Fine della configurazione dell'ambiente di lavoro e dell'infrastruttura
			\item Installazione della suite Atlassian
		\end{itemize}
		\item \textbf{Terza Settimana (24 ore)} 
		\begin{itemize}
			\item Installazione della suite Atlassian
			\item Configurazione ed integrazione dei prodotti
		\end{itemize}
		\item \textbf{Quarta Settimana (36 ore)} 
		\begin{itemize}
			\item Configurazione ed integrazione dei prodotti

		\end{itemize}
		\item \textbf{Quinta Settimana (40 ore)} 
		\begin{itemize}
			\item Test con dati di mock
		\end{itemize}
		\item \textbf{Sesta Settimana (40 ore)} 
		\begin{itemize}
			\item Test da parte degli utenti
			\item Fine tuning del prodotto
		\end{itemize}
		\item \textbf{Settima Settimana (40 ore)} 
		\begin{itemize}
			\item Test da parte degli utenti
			\item Sviluppo della documentazione
			\item Fine tuning del prodotto
		\end{itemize}
		\item \textbf{Ottava Settimana (40 ore)} 
		\begin{itemize}
			\item Test da parte degli utenti
			\item Sviluppo della documentazione
		\end{itemize}
		\item \textbf{Nona Settimana (40 ore)} 
		\begin{itemize}
			\item Test da parte degli utenti
			\item Sviluppo della documentazione
			\item Eventuale migrazione in ambiente di produzione
		\end{itemize}
		\item \textbf{Decima Settimana (40 ore)} 
		\begin{itemize}
			\item Eventuale migrazione in ambiente di produzione
			\item Presentazione del lavoro
		\end{itemize}
    \end{itemize}
}

% Indicare il totale complessivo (deve essere compreso tra le 300 e le 320 ore)
\newcommand{\totaleOre}{}

\newcommand{\obiettiviObbligatori}{
	 \item \underline{\textit{O01}}: tutti i software del nuovo sistema devono essere correttamente configurati;
 	 \item \underline{\textit{O02}}: tutti i software del nuovo sistema devono essere in grado di comunicare tra loro;
	 \item \underline{\textit{O03}}: deve essere presente una documentazione chiara ed appropriata sia per chi utilizza il software sia per chi lo andrà a mantenere;
	 
}

\newcommand{\obiettiviDesiderabili}{
	 \item \underline{\textit{D01}}: possibilità di integrazione con la suite di prodotti Microsoft Office;
	 \item \underline{\textit{D02}}: possibilità di effettuare personalizzazioni tramite l'utilizzo di script che andranno ad interagire con i software che compongono il sistema;
}

\newcommand{\obiettiviFacoltativi}{
	 \item \underline{\textit{F01}}: possibilità di migrazione dei dati relativi a profetti già in produzione sul nuovo sistema;
	 \item \underline{\textit{F02}}: possibilità di trasportare il sistema in ambiente di produzione rimpiazzando (anche in maniera graduale) quello già esistente;
}