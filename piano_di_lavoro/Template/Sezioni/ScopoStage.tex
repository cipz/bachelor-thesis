%----------------------------------------------------------------------------------------
%	STAGE DESCRIPTION
%----------------------------------------------------------------------------------------
\section*{Scopo dello stage}
% Personalizzare inserendo lo scopo dello stage, cioè una breve descrizione
Lo studente avrà quindi il compito di configurare i software scelti dall'azienda, testare, in caso funzioni correttamente spostare in ambiente di sviluppo


L'obiettivo di questo progetto di stage è la configurazione di un ambiente per la gestione dei workflow di progetti aziendali tramite l'utilizzo dei software della suite di Atlassian.
Questi permettono agli utenti dell'azienda di gestire in maniera centralizzata non solo i task di un singolo progetto ma di avere un pannello di controllo che permetta loro di vedere i progressi di più progetti in corso, insieme allo storico relativo alla loro documentazione e i collegamenti alla repository.
La suite di tool e di plugin richiesti da Athonet per l'installazione da parte dello stagista è composta da:
\begin{itemize}
	\item Jira Software
	\item Portfolio for Jira
	\item Jira Service Desk
	\item Confluence
	\item Bitbucket
\end{itemize}
Lo studente avrà quindi il compito di installare e configurare tali software su richiesta dell'azienda, effettuare test insieme a chi andrà ad interfacciarcisi, creare la documentazione adeguata ed effettuare un eventuale porting da ambiente di produzione ad amviente di sviluppo.
